\section{Requisitos del Sistema}

% ... (contenido previo de requisitos y estudio de usuarios)

\section{Metodología de Desarrollo}

El desarrollo de este proyecto se ha basado en una metodología iterativa y centrada en el usuario, combinando la exploración de alternativas técnicas con la validación incremental de prototipos. El proceso se ha estructurado en varias fases, cada una con objetivos y entregables definidos, permitiendo adaptar la solución a los retos técnicos y a las necesidades detectadas en los usuarios finales.

\subsection{Fases del Proyecto}
\begin{itemize}
    \item \textbf{1. Análisis de requisitos y estudio de usuarios:} Se identificaron los requisitos funcionales y no funcionales mediante revisión bibliográfica, análisis de casos de uso y entrevistas con usuarios potenciales (profesionales de la salud, cuidadores y usuarios finales).
    \item \textbf{2. Revisión de tecnologías y estándares:} Se realizó un análisis comparativo de las tecnologías disponibles para la integración de datos de wearables, así como de los principales estándares de interoperabilidad en salud digital.
    \item \textbf{3. Prototipado y validación de alternativas:} Se desarrollaron prototipos para evaluar la viabilidad de las distintas opciones técnicas (integración de APIs, modelos de datos, almacenamiento, visualización, etc.), seleccionando las soluciones más adecuadas en base a pruebas y criterios de escalabilidad, seguridad y facilidad de uso.
    \item \textbf{4. Implementación incremental:} El sistema se desarrolló de forma modular, permitiendo la integración progresiva de los distintos componentes (adquisición de datos, backend, base de datos, frontend, etc.) y la validación continua de su funcionamiento.
    \item \textbf{5. Pruebas y ajustes:} Se llevaron a cabo pruebas funcionales, de integración y de usuario para asegurar el cumplimiento de los requisitos y la calidad del sistema. Los resultados de estas pruebas guiaron los ajustes y mejoras finales.
    \item \textbf{6. Documentación y entrega:} Finalmente, se elaboró la documentación técnica y de usuario, y se preparó la memoria del proyecto.
\end{itemize}

\subsection{Gestión del Tiempo}

A lo largo del desarrollo del proyecto, se ha realizado una planificación temporal detallada, distribuyendo las tareas en función de su prioridad y complejidad. En la Tabla~\ref{tab:planificacion} se muestra la planificación temporal estimada para cada fase del proyecto. (\textit{Rellenar con el tiempo dedicado a cada fase una vez finalizado el proyecto.})

\begin{table}[H]
    \centering
    \begin{tabular}{|l|c|}
        \hline
        \textbf{Fase} & \textbf{Tiempo dedicado (horas)} \\
        \hline
        Análisis de requisitos y estudio de usuarios & \\
        Revisión de tecnologías y estándares & \\
        Prototipado y validación de alternativas & \\
        Implementación incremental & \\
        Pruebas y ajustes & \\
        Documentación y entrega & \\
        \hline
        \textbf{Total} & \\
        \hline
    \end{tabular}
    \caption{Planificación temporal del proyecto}
    \label{tab:planificacion}
\end{table}
